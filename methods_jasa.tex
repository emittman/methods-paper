%&latex
\documentclass[12pt]{article}
\usepackage{amsmath,amsfonts,bm}
\usepackage{graphicx,psfrag,epsf}
\usepackage{enumerate}
\usepackage{natbib}
\usepackage{url} % not crucial - just used below for the URL 
\usepackage{etoolbox}
\usepackage{booktabs}
\usepackage{multirow}
\usepackage{lscape}
%\pdfminorversion=4
% NOTE: To produce blinded version, replace "0" with "1" below.
\newcommand{\blind}{0}

% DON'T change margins - should be 1 inch all around.
\addtolength{\oddsidemargin}{-.5in}%
\addtolength{\evensidemargin}{-.5in}%
\addtolength{\textwidth}{1in}%
\addtolength{\textheight}{1.3in}%1.3?
\addtolength{\topmargin}{-.8in}%
\graphicspath{{figures_tables/}}

\newcommand{\op}{\operatorname}
\newcommand{\ind}{\stackrel{ind.}{\sim}}

\begin{document}
\newtoggle{thesis}
\toggletrue{thesis}
% \bibliographystyle{natbib}

\def\spacingset#1{\renewcommand{\baselinestretch}%
{#1}\small\normalsize} \spacingset{1}


%%%%%%%%%%%%%%%%%%%%%%%%%%%%%%%%%%%%%%%%%%%%%%%%%%%%%%%%%%%%%%%%%%%%%%%%%%%%%%

\if1\blind
{
  \title{\bf A Bayesian nonparametric analysis of RNA-seq gene expression data}
  \author{Author 1\thanks{
    The author gratefully acknowledges Dr. Jarad Niemi for advise and comments.\textit{This project was funded in part by a grant from the NIH, XXXX.}}\hspace{.2cm}\\
    Department of Statistics, Iowa State University}
  \maketitle
} \fi

\if0\blind
{
  \bigskip
  \bigskip
  \bigskip
  \begin{center}
    {\LARGE\bf A Bayesian nonparametric analysis of RNA-seq gene expression data}
\end{center}
  \medskip
} \fi

\bigskip
\begin{abstract}
We present a Bayesian nonparametric model suitable for gene expression profiling using high-throughput sequencing technologies. An analysis pipeline is detailed for the analysis of RNA-seq data and we apply it to an experiment done on several maize genotypes. This case study highlights an advantage that Bayesian methods enjoy --- that evaluating the probability of arbitrary hypotheses regarding the model parameters is straightforward. Simulations studies are conducted which suggest that our method provides superior parameter estimates and is able to improve the accuracy of gene classification compared to other, popular methods. To make our method computationally feasible we implemented a blocked Gibbs sampler exploiting fine grain parallelism on general purpose graphical processing units.

\end{abstract}

\noindent%
{\it Keywords:}  Dirichlet process, hierarchical modeling, parallel computing
\vfill

\newpage
\tableofcontents

\spacingset{1.45} % DON'T change the spacing!


\section{Introduction}
Because there are many genes to observe, a gene expression experiment presents an opportunity to learn the underlying distribution of the gene-specific parameters. Choosing to model this underlying distribution with a parametric family that is too restrictive, i.e. leads to too crude an approximation of the underlying process, means that very little borrowing of information across genes.

We also want to estimate functionals of the parameters by integrating over the entire posterior.

\section{Bayesian nonparametric model for gene expression}

Model goes here

\section{Computation}

\section{Voom}
We'd like to know whether it makes a difference whether we use voom precision weights. It might be good to compare results with and without the weights for the non-parametric method vs a standard method, like vanilla limma.

\section{Analysis of heterosis data}




\appendix
\include{appendix}
\bigskip
\begin{center}
{\large\bf SUPPLEMENTARY MATERIAL}
\end{center}

\begin{description}

\item[Title:] Brief description. (file type)

\item[R-package for  MYNEW routine:] R-package ?MYNEW? containing code to perform the diagnostic methods described in the article. The package also contains all datasets used as examples in the article. (GNU zipped tar file)

\item[HIV data set:] Data set used in the illustration of MYNEW method in Section~ 3.2. (.txt file)

\end{description}


\bibliographystyle{plainnat}

\bibliography{methods}
\end{document}
