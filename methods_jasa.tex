%&latex
\documentclass[12pt]{article}
\usepackage{amsmath,amsfonts,bm}
\usepackage{graphicx,psfrag,epsf}
\usepackage{enumerate}
\usepackage{natbib}
\usepackage{url} % not crucial - just used below for the URL 
\usepackage{etoolbox}
\usepackage{booktabs}
\usepackage{multirow}
\usepackage{lscape}
\usepackage{blkarray}
%\pdfminorversion=4
% NOTE: To produce blinded version, replace "0" with "1" below.
\newcommand{\blind}{0}

% DON'T change margins - should be 1 inch all around.
\addtolength{\oddsidemargin}{-.5in}%
\addtolength{\evensidemargin}{-.5in}%
\addtolength{\textwidth}{1in}%
\addtolength{\textheight}{1.3in}%1.3?
\addtolength{\topmargin}{-.8in}%
\graphicspath{{figures_tables/}}

\newcommand{\op}{\operatorname}
\newcommand{\ind}{\stackrel{ind}{\sim}}

\begin{document}
\newtoggle{thesis}
\toggletrue{thesis}
% \bibliographystyle{natbib}

\def\spacingset#1{\renewcommand{\baselinestretch}%
{#1}\small\normalsize} \spacingset{1}

\setcitestyle{round}
%%%%%%%%%%%%%%%%%%%%%%%%%%%%%%%%%%%%%%%%%%%%%%%%%%%%%%%%%%%%%%%%%%%%%%%%%%%%%%

\if1\blind
{
  \title{\bf Detection of gene heterosis in Maize using a Bayesian nonparametric model}
  \author{Eric Mittman\thanks{Department of Statistics, Iowa State University}\\
    and\\
    Jarad Niemi\footnotemark[1]}
  \maketitle
} \fi

\if0\blind
{
  \bigskip
  \bigskip
  \bigskip
  \begin{center}
    {\LARGE\bf Detection of gene heterosis in maize using a Bayesian nonparametric model}
\end{center}
  \medskip
} \fi

\bigskip
\begin{abstract}
Heterosis, or hybrid vigor, refers to biological differences seen in offspring of two inbred parents, which give the hybrid qualities superior to either parent. While the phenomenon is well-known, in many cases genetic causes have yet to be determined. We consider RNA-seq data obtained in an experiment conducted on four varieties of maize plants with the goal of using measurements of gene expression to categorize and identify gene heterosis which might serve as a genetic basis for the heterosis. We present a Bayesian nonparametric model suitable for the analysis of gene expression profiling data. By learning the underlying distribution of the gene-specific parameters, our model allows for appropriate borrowing of information across genes to improve gene-specific inference. Furthermore, our fully Bayesian inference allows for straightforward computation of posterior probabilities for a large class of hypotheses, like gene expression heterosis. Simulations studies provide evidence that our method provides improved parameter estimates over competing methods and that it may improve the accuracy of gene classification. We compare the results of our analysis of the Paschold data set with those of Landau, who assumed a parameteric distribution for the gene specific parameters.

\end{abstract}

\noindent%
{\it Keywords:}  gene expression, Dirichlet process, hierarchical modeling, parallel computing
\vfill

\newpage
\tableofcontents

\spacingset{1.45} % DON'T change the spacing!


\section{Introduction}
Because there are many genes to observe, a gene expression experiment presents an opportunity to learn the underlying distribution of the gene-specific parameters. Choosing to model this underlying distribution with a parametric family that is too restrictive, i.e. leads to too crude an approximation of the underlying process, means that very little borrowing of information across genes.

We also want to estimate functionals of the parameters by integrating over the entire posterior.

\section{Bayesian nonparametric model for gene expression}

Model goes here

\section{Computation}

\section{Voom}
We'd like to know whether it makes a difference whether we use voom precision weights. It might be good to compare results with and without the weights for the non-parametric method vs a standard method, like vanilla limma.

\section{Analysis of heterosis data}




\appendix
\include{appendix}
% \bigskip
% \begin{center}
% {\large\bf SUPPLEMENTARY MATERIAL}
% \end{center}

% \begin{description}
% 
% \item[Title:] Brief description. (file type)
% 
% \item[R-package for  MYNEW routine:] R-package ?MYNEW? containing code to perform the diagnostic methods described in the article. The package also contains all datasets used as examples in the article. (GNU zipped tar file)
% 
% \item[HIV data set:] Data set used in the illustration of MYNEW method in Section~ 3.2. (.txt file)
% 
% \end{description}


\bibliographystyle{abbrvnat}

\bibliography{methods}
\end{document}
